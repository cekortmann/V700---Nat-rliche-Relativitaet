\section{Aufbau}
\label{sec:Aufbau}

Der Versuchsaufbau besteht aus einem Geiger-Müller-Zählrohr, welches ein großes 
Detektionsfenster besitzt, aber die Strahlungsart nicht unterscheiden kann. 
Um diese zu unterscheiden wird das Absorptionsvermögen mittels der Schutzkappe des 
Detektionsfensters und einer Aluplatte von geringer Schichtdicke. Das Geiger-Müller-Zählrohr 
muss aufgrund der geringen Reichweite von Alphastrahlung möglichst nah an der Probe platziert werden.
Dazu dient ein Gestell. Desweiteren sind Ballons sowie Paranüsse und Blaukorn als Proben bereit gelegt.