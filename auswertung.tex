\section{Auswertung}
\label{sec:Auswertung}

%\subsection{Fehlerrechnung}
\label{sec:Fehlerrechnung}
Für die Fehlerrechnung werden folgende Formeln aus der Vorlesung verwendet.
für den Mittelwert gilt
\begin{equation}
    \overline{x}=\frac{1}{N}\sum_{i=1}^N x_i ß\; \;\text{mit der Anzahl N und den Messwerten x} 
    \label{eqn:Mittelwert}
\end{equation}
Der Fehler für den Mittelwert lässt sich gemäß
\begin{equation}
    \increment \overline{x}=\frac{1}{\sqrt{N}}\sqrt{\frac{1}{N-1}\sum_{i=1}^N(x_i-\overline{x})^2}
    \label{eqn:FehlerMittelwert}
\end{equation}
berechnen.
Wenn im weiteren Verlauf der Berechnung mit der fehlerhaften Größe gerechnet wird, kann der Fehler der folgenden Größe
mittels Gaußscher Fehlerfortpflanzung berechnet werden. Die Formel hierfür ist
\begin{equation}
    \increment f= \sqrt{\sum_{i=1}^N\left(\frac{\partial f}{\partial x_i}\right)^2\cdot(\increment x_i)^2}.
    \label{eqn:GaussMittelwert}
\end{equation}
Alle Messwerte werden im folgenden für ein Zeitintervall von einer Sekunde angegeben, sodass diese direkt die Aktivität der Probe darstellen.

Die gemessene Untergrundrate ergibt sich zu
\begin{equation*}
    5.8167 \pm 2.4118 \,\unit{\becquerel} \; .
\end{equation*}
Dabei wird die Standardabweichung über 
\begin{equation}
    \sigma = \sqrt{N}
    \label{eqn:standardabweichung}
\end{equation}
berechnet. 
Die Hintergrundrate muss von allen weiteren Messungen abgezogen werden.

\subsection{Natürliche radioaktive Stoffe}
In der ersten Messung wird die Aktivität von Paranüssen, dessen Masse $m = 17.45 \, \unit{\gram}$ beträgt, bestimmt. Dazu wurde fünfmal für das Zeitintervall 
von einer Minute gemessen. Wenn alle Messwerte aufaddiert und dann für das Intervall von einer Sekunde angegeben werden, ergibt sich 
\begin{equation*}
    7.4667 \pm 2.7325 \, \unit{\becquerel} \; .
\end{equation*}
Von diesem  Wert muss nun noch die Hintergrundstrahlung abgezogen werden, sodass sich die Aktivität von Paranüssen zu 
\begin{equation*}
    A_{\symup{Paranuss}} = 1.65 \pm 1.28 \, \unit{\becquerel} \; . 
\end{equation*}
