\section{Diskussion}
\label{sec:Diskussion}

Die in \autoref{sec:stoffe} berechneten spezifischen Aktivitäten werden nun mit theoretischen Werten verglichen. Für Paranüsse ergibt sich für die spezifische 
Aktivität ein Wert zwischen $8\,\unit{\becquerel\per\kilo\gram}$ und $130\,\unit{\becquerel\per\kilo\gram}$ \cite{ap700}. Der experimentell bestimmte Wert 
$94.56\,\unit{\becquerel\per\kilo\gram}$ fällt genau in dieses Intervall.

Zu der spezifischen Aktivität von Blaukorn konnte kein Literaturwert gefunden werden, sodass ein Vergleich ausbleibt. Es fällt aber auf, dass die 
spezifische Aktivität von Blaukorn wesentlich höher ist als die von Paranüssen. 

In Bezug auf den Strahlenschutz lässt sich mit Hilfe  der spezifischen Aktivitäten keine Aussage zu dem Schaden für den Menschen durch die beiden oben genannten 
Stoffe treffen. Eine Gewichtung der Strahlenarten ist nötig. Beim Blaukorn ist auffällig, dass dieses vorwiegend aus Alphastrahlung besteht, sodass eine 
Gefährdung nur durch Verschlucken oder Einatmen entstehen kann, da die Haut die radioaktive Strahlung bereits abschirmt und in den meisten Fällen auch bereits 
durch die Luft zwischen Blaukorn und Mensch. Die übrige Strahlung ist wiederum so gering, dass sie kaum einen Einfluss auf den Menschen hat. Das Gleiche 
gilt für die Paranüsse. Diese bestehen nur zu 26 Prozent uas Alphastrahlung, sodass die meiste Strahlung nicht durch die Haut abgeschirmt werden kann. 
Jedoch ist die gesamte spezifische Aktivität so gering, dass eine größere Masse gegessen werden müsste, um Schaden von Paranüssen zu vernehmen.