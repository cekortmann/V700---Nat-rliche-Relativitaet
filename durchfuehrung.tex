\section{Durchführung}
\label{sec:Durchführung}

Zuerst wird die Untergrundrate an dem Messplatz gemessen. Dazu wird das Geiger-Müller-Zählrohr auf dem 
Gestell platziert und die Zählrate wird für ein Zeitintervall von 10 Minuten gemessen.
Das Zeitintervall lässt sich direkt am Geiger-Müller-Zählrohr mittels der Funktionstasten COUNT, MENU sowie Plus und Minus einstellen.

Als nächstes wird die Zerfallsrate von zwei verschiedenen Proben gemessen. Es muss die Masse 
der Probe notiert werden. Zur Messung der Rate der Paranüsse wird ein Zeitintervall von einer Minute gewählt. 
Diese Messung wird fünfmal wiederholt. Die Messung des Blaukorn wird über ein Intervall von 
zehn Minuten durchgeführt. Bei beiden Proben werden jeweils einminütige Messungen mit der Schutzkappe und mit 
der Aluminiumplatte durchgeführt, sodass sich die Art der Strahlung sowie der Anteil der 
Alphastrahlung bestimmen lässt.

Zuletzt wird die Radioaktivität der Raumluft gemessen. Dazu muss zunächst die Untergrundstrahlung eines 
Luftballons gemessen werden. Dies wird fünfmal für jeweils ein Zeitintervall von einer Minute getan. Im Anschluss 
wird der Luftballon aufgeblasen und mit Hilfe eines Fleecestoffs elektrostatisch aufgeladen. Dann wird der Ballon für 20 Minuten frei 
schwebend in den Keller gehangen. Nach dieser Zeit wird die Luft wieder aus dem Ballon herausgelassen. Jetzt kann die Zerfallsrate 
über ein Zeitintervall von 5 Minuten gemessen werden. Auch hier wird mittels der Schutzkappe und der Aluminiumplatte die Art des Zerfalls sowie 
der Anteil der Alphastrahlung bestimmt. 