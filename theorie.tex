\section{Theorie}
\label{sec:Theorie}

Ein Atom setzt sich aus einem Kern, welcher wiederum aus Protonen und Neutronen zusammengesetzt ist, und Elektronen.
Bei einem neutralen Atom gleicht die Anzahl der Protonen die Anzahl der Hüllenelektronen. Bei instabilen Nukliden werden die Kerne infolge eines radioaktiven
Zerfalls in einen anderen Zustand oder einem anderen Kern umgewandelt. Insgesamt kann in drei verschiedene radioaktive Zerfälle unterschieden werden.
Sowohl beim $\alpha$- als auch beim $\beta$-Zerfall werden die Kerne durch den Zerfall veränder, bei der $\gamma$-Strahlung hingegen handelt es sich um eine Änderung
des Atomzustands.

Die $\alpha$-Strahlung ist nicht anderes als ein Heliumkern, welcher aus dem Elteratom ausgestoßen wurde.
Es folgt somit
\begin{equation*}
    { }_Z^A X \longrightarrow{ }_{Z-2}^{A-4}+{ }_2^4He
\end{equation*}
wobei $ A$ für die Massenzahl (Ordnungszahl + Anzahl der Neutronen) und $Z$ für die Anzahl der Neutronen steht. $X$ ist der Elterkern und $Y$ der Tochterkern

Die $\beta$-Strahlung kann in zwei weitere Abschnitte unterteilt werden. So geht bei der $\beta^-$-Strahlung ein NEutron in ein Proton über und bei der $\beta^+$-Strahlung umgekehrt.
Für den $\beta^+$-Zerfall folgt daraus
\begin{equation*}
    { }_Z^A X \longrightarrow{ }_{Z-1}^A Y+\beta^{+}+\nu_e
\end{equation*}

Die $\gamma$-Strahlung hat zur Folge, dass durch Emission von elektromagnetischer Strahlung der Atomkern von einem angeregten Zustand in einen tieferen abfällt. Damit ein Kern in einen
angeregten Zustand vorliegen kann, muss dem Kern einer der beiden anderen Zerfälle zuvor widerfahren sein.

Der radioaktive Zerfall ist ein statistischer Prozess, dadurch kann es zu Schwankungen und Ungenauigkeiten der Zerfallsdauer und Aktivität kommen.
Radioaktive Zerfälle unterliegen der Poisson-Verteilung. Die einzelnen Nuklide besitzen eine unterschiedliche mittlere Lebensdauer $\tau $, mit welcher wiederum die Größe $\lambda= \tau^{-1}$ bestimmt werden kann.
Die Größe $\lambda $ wird auch als Zerfallskonstante bezeichnet.
Mit diesen Kenngrößen kann das Zerfallsgesetz aufgestellt werden
\begin{equation*}
    N(t)= N_0 e^{-\lambda t} \, ,
\end{equation*}
wobei $N_0$ die Anzahl der anfangs vorhandenen Kerne ist und $N(t)$ die Anzahl der Kerne zum Zeitpunkt $t$ beschriebt.
Um die Zerfälle besser einschätzen zu können, kann die Halbwärtszeit eingeführt werden
\begin{equation*}
    T_{1/2}= \frac{\text{2}}{\lambda}\, . 
\end{equation*} 
Die Intensität der Zerfälle von Radionukliden wird durch die Aktivität $A$ angegeben
\begin{equation}
    A=\frac{N}{\Delta t}\, .
    \label{eqn:Aktivität}
\end{equation}
Die Aktivität gibt die radioaktiven Zerfälle $N$ pro Zeit $\Delta t$ an und hat die Einheit Bequerel (Bq = 1/s).
Darauf aufbauend kann die spezifische Aktivität definiert werden, welche die Aktivität pro Masse angibt.  